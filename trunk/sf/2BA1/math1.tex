\documentclass[a4paper,12pt]{article}
\begin{document}

\section{Question 1}

Prove that$(3n)! \leq 27(n!)^3$ for all $n \in N$

The formula holds for n=1.
Suppose the formula holds for $m=n$ where m is some natural number.
\[ (3m)! \leq 27(m!)^3 \]
Then \[(3(m + 1))! \leq 27((m + 1)!)^3\]
\[(3m)!(3m + 3)(3m + 2)(3m + 1) \leq 27{((m!)(m + 1))}^3\]
\[(3m)!(3m + 3)(3m + 2)(3m + 1) \leq 27({(m!)}^3{(m + 1)}^3)\]
\\*
But \[(3m)! \leq 27(m!)^3\]
\\*
Which leaves \[(3m + 3)(3m + 2)(3m + 1) \leq 27{(m + 1)}^3\]
\[27{m}^3 + 54{m}^2 + 33m + 6 \leq 27{m}^3 + 81{m}^2 + 81m + 27\]
Which holds for all $m \in N$.
It now follows from the Principle of Mathematical Induction that this identity holds for all natural numbers m.

\section{Question 2}
Prove that \[(A \cup B) \backslash (A \cap B) = (A \backslash B) \cup (B \backslash A)\]
Let $x \in (A \cup B)\backslash(A \cap B)$. Then $x \in A \cup B, x \notin A \cap B$
Then $x \in A$ or $x \in B$ but $x \notin A \cap B$.
\\*
If $x \in A$, then $x \notin B$, $x \in A \backslash B$. If $x \in B$, then $x \notin A$, $x \in B \backslash A$.
Then if $x \in A$, then $x \notin B$ and $x \notin B \backslash A$. Similarly for $x \in B$.
Thus $x \in (A \backslash B) \cap (B \backslash A)$
\\*
Now let $x \in (A \backslash B) \cup (B \backslash A)$. Then either $x \in A \backslash B$ or $x \in B \backslash A$. 
In both cases $x \in A \cup B$, $x \notin A \cap B$. Thus $x \in (A \backslash B) \cup (B \backslash A)$.
\\*
\\
We have shown that the sets $(A \cup B) \backslash (A \cap B)$ and $(A \backslash B) \cup (B \backslash A)$
have the same elements, and thus that $(A \cup B) \backslash (A \cap B) = (A \backslash B) \cup (B \backslash A)$.

\section{Question 3}
\subsection{Relation Q on N}
The relation $Q$ on $N$ is not reflexive as \[(1)^2 - (1)^2 = 2^k\]
\[1 - 1 = 2^k\]
\[0 = 2^k\]
There is no $k \in N$ such that $2^k = 0$.
\\*
\\
The relation $Q$ on $N$ is not symmetric. \[(3)^2 - (1)^2 = 2^k\]
\[9 -1 = 2^k\]
\[8 = 2^k\]
\[8 = 2^3\]
But \[(1)^2 - (3)^2 = 2^k\]
\[1 - 9 = 2^k\]
\[ -8 = 2^k \]
\\*
There exists no $k \in N$ such that $2^k = -8$.
\\*
\\
The relation $Q$ is not antisymmetric. $1Q3$ and $3Q1$ but $ 1 \not= 3$
\\* 
\\
The relation $Q$ is not transitive. $10Q6$ and $6Q2$ but $10\not Q 2$
\\*
\\
As the relation $Q$ is not reflexive, symmetric, anti-symmetric or transitive it is not an equivalence 
relation or a partial order.
\subsection{Relation R on N}
The relation $R$ on $N$ is reflexive as \[n^2 / n^2 = 2^k\]
Where $n \in N$,$n > 0$
\[1 = 2^k\]
\[1 = 2^0\]
\\*
\\
The relation $R$ is not symmetric as \[ 2^2 / 1^2 = 2^k\]
\[4 / 1 = 2^k\]
\[ 4 = 2^2\]
But \[ 1^2 / 2^2 = 2^k\]
\[ 1 / 4 = 2^k\]
There exists no element $k \in N$ such that $2^k = 1/4$.
\\*
\\
Let $x,y \in N$, $x,y > 0$ satisfy $xRy$. Then $x^2 / y^2 = 2^k$ for some $k \in N$.
Then $y^2 / x^2 = 2^{-k}$, but $-k \notin N$. Therefore the realtion is not anti-symmetric.


\end{document}
